\documentclass[a4paper,12pt]{report}

\usepackage{amsmath}
\usepackage{graphicx}
\usepackage{float}
\usepackage[hidelinks]{hyperref} 
\usepackage{fancyhdr}

% Configuration de l'en-tête
\pagestyle{fancy}
\fancyhf{}
\fancyhead[R]{\includegraphics[width=2cm]{images/logoCY.png}}

\begin{document}
	
	% Page de garde
	\begin{titlepage}
		
		\centering
		\vspace*{2cm}
		\Huge{\textbf{Projet Développement Web}} \\[1cm]
		\Large{Professeur : Mahdi Mariem} \\[2cm]
		\vfill
		\Large{
			Classe : ING 1 GIA 2 \\
			Auteur : Balit Ilian, Éléonore GEORGEL, Inès RIBAR, Myriam SAADI, Zerrouki Abbes \\
			Date : \today \\
		}
	\end{titlepage}
	
	% Sommaire
	\tableofcontents
	\newpage
	
	\section{Introduction et Contexte}
	
	\subsection{Présentation générale du projet}
	
	Ce projet s'inscrit dans le cadre d'un projet scolaire visant à concevoir et développer une plateforme numérique pour une ville intelligente (dans notre cas). Cette plateforme cherche à répondre aux besoins potentiels des utilisateurs d'une ville intelligente en apportant des solutions numériques nouvelles et accessibles. Dans un monde où les technologies évoluent de plus en plus rapidement, les villes et villages sont de plus en plus amené à s'adapter aux attentes croissantes des citoyens, notamment en termes de services, de connectivité, et de gestion des ressources.
	
	L’objectif principal de ce projet est de créer une solution numérique intégrée qui regroupe une variété d'objets connectés, de services et de fonctionnalités. Tout ces outils ont pour objectif d'améliorer la qualité de vie des citoyens en facilitant leur accès à l’information, à des services locaux, et aux objets connectés de la ville. La plateforme est dans un premier temps pensée comme un moyen d'accès à l'information assez simple. Ensuite comme un moyen de regrouper, gérer et contrôler les objets connectés, services et outils d'une ville.
	\subsection{Objectifs}
	
	Ce projet a plusieurs objectifs qui sont bien définis :
	
	\begin{itemize}
		\item \textbf{Centraliser l'information locale :} Pouvoir regrouper toutes les informations pertinentes, telles que les événements locaux, les lieux d'intérêts et les horaires des transports, afin de faciliter l’accès des citoyens à ces données.
		\item \textbf{Faciliter la gestion des outils/services publics :} Offrir une plateforme unique permettant de gérer et accéder à des services partagés comme des outils ou équipements publics.
		\item \textbf{Permettre une gestion centralisée et efficace :} Les administrateurs auront la possibilité de superviser et de gérer l’ensemble du système : gestion des utilisateurs, des services/outils et des objets connectés.
		\item \textbf{Optimiser l'expérience utilisateur :} Mettre en place une interface moderne, intuitive et agréable qui facilite la navigation et l’interaction de l'utilisateur avec la plateforme, et qui répond aux attentes de la ville ainsi que des utilisateurs en termes de fonctionnalités et d’accessibilité.
	\end{itemize}
	
	\subsection{Enjeux}
	
	Dans une ville intelligente, l'intégration des technologies au quotidien devient essentielle pour offrir aux utilisateurs de la plateforme des services et outils efficaces et faciles d'accès. Ce projet se base sur cette vision en cherchant à répondre à plusieurs enjeux importants :
	
	\begin{itemize}
		\item \textbf{Accessibilité à l’information locale :} Les citoyens de cette ville auront besoin d’un accès rapide et facile aux informations locales telles que les événements, les horaires des transports publics et les principaux lieux d'intérêts de la ville. Cette plateforme vise à réunir au même endroit toutes ces informations afin de les rendre accessibles à tout moment pour chaque utilisateur de la plateforme.
		\item \textbf{Optimisation des services et outils :} Une autre dimension du projet est de faciliter l’interaction des citoyens avec les différents services et outils de la ville. Que ce soit pour la gestion des outils partagés, ou d'équipements spécifiques, la plateforme permet une gestion simplifiée et efficace.
		\item \textbf{Gestion des objets connectés et des utilisateurs :} Un des objectifs clés du projet est de permettre aux administrateurs de gérer sur cette plateforme les différents éléments de cette ville intelligence. Cela inclut la gestion des utilisateurs, des services/outils offerts, et des objets connectées présent dans la ville, tout en garantissant la sécurité des données et la confidentialité des informations.
	\end{itemize}
	
	Ainsi, cette plateforme numérique vise à simplifier et enrichir la vie quotidienne des citoyens en intégrant des technologies intelligentes et en leur offrant un accès direct à une multitude de services/outils et d'informations. L’optimisation des processus administratifs et l'amélioration de la gestion des ressources sont également au cœur de ce projet, qui a pour ambition de créer un environnement plus connecté, plus efficace, et plus réactif aux besoins des citoyens d'une ville connectée.
	
	\subsection{Choix du thème du projet}
	
	Le choix du thème de ce projet, centré sur le développement d'une plateforme numérique intelligente pour une ville intelligente, a été guidé par plusieurs facteurs pertinents, à la fois scolaires, technologiques et sociétaux.
	
	\begin{itemize}
		\item \textbf{1. Contexte scolaire} Le choix s'est fait entre nous, membres du groupe, pour plusieurs raisons. Dans un premier temps le sujet de la ville intelligente est celui que nous avons à la majorité trouvé comme étant le plus intéressant nottament pour les raisons technologiques et sociétales comme évoque ci-dessous. Puis dans un second temps ce choix de sujet est celui dans lequel nous arrivions le plus à nous projeter vis-a vis du fond et de la forme que nous comptions donner à la plateforme. Il s'agit du sujet qui en totalité nous semblait le plus intéressant.
		\item \textbf{2. Contexte technologique actuel} Dans un monde où les technologies évoluent rapidement, les villes et villages intelligents représentent une évolution majeure de l'urbanisme et de la gestion des espaces publics. L'essor des technologies numériques, l'Internet des objets (IoT) et la gestion intelligente des données offrent de nouvelles possibilités pour améliorer la qualité de vie des citoyens. En choisissant ce thème, nous avons voulu explorer les enjeux de l'intégration de ces technologies dans le quotidien des habitants d'une ville connectée, en utilisant des outils modernes comme les bases de données (MySQL), un frameworks web (Symfony) et les langages de programmation tels que HTML, CSS, JavaScript.
		\item \textbf{3. L'importance de l'innovation dans le domaine de l'urbanisme} 	Le thème choisi s'inscrit également dans une tendance plus large d'innovation dans le domaine de l'urbanisme et du développement des villes intelligentes. En proposant une solution intégrée qui centralise diverses fonctionnalités telles que l'accès à l'information locale, la gestion des services publics et la possibilité d'interagir avec différents objets et outils, ce projet contribue à une réflexion sur l'avenir des villes connectées et intelligentes. Ce choix permet aussi de traiter des enjeux environnementaux, économiques et sociaux, en rendant les services urbains plus efficaces, durables et accessibles.
	\end{itemize}	
	
	\subsection{Outils et technologies choisis}
	
	Pour le développement de cette plateforme, plusieurs technologies ont été employées :
	
	\begin{itemize}
		\item  HTML et CSS pour la structure et la mise en forme du site.
		\item JavaScript pour l'interactivité et la gestion dynamique des éléments sur la page.
		\item Symfony, un framework PHP, a été utilisé pour la gestion des fonctionnalités côté serveur.
		\item Une base de données pour stocker et gérer les informations des utilisateurs, objets, informations, et autres données essentielles à la plateforme.
	\end{itemize}
	
	\newpage
	
	\section{Répartition des tâches entre les membres du groupe}
	
	\subsection{Répartition des responsabilités}
	% Utiliser ici un graphique du type Gantt
	
	\subsection{Contributions de chaque membres}
	
	\newpage
	
	\section{Étapes réalisées}
	
	Le projet s'est déroulé en plusieurs étapes structurées, permettant une progression méthodique depuis la phase de planification jusqu'à l'obtention des premiers résultats concrets.
	
	\subsection{Phase de planification}
	La première étape a consisté à définir précisément les objectifs du projet, en accord avec les attentes du projet. Nous avons ensuite réalisé plusieurs réunions au préalable pour identifier les fonctionnalités principales à intégrer dans notre plateforme numérique intelligente.
	
	À partir de cette analyse, nous avons structuré notre projet autour des quatre modules principaux : Information, Visualisation, Gestion et Administration. Un planning global autour de ces modules a été établi afin d'organiser les tâches et répartir le travail entre les membres du groupe, en tenant compte des priorités et des délais impartis.
	
	\subsection{Phase de conception}
	Durant la phase de conception, nous avons travaillé sur l'architecture de la plateforme et la définition de ses principales composantes. Nous avons conçu au cours du projet ce qui devint la maquette visuelle de notre site, nous avons réfléchi à l'ergonomie générale du site et élaboré la structure de la base de données nécessaire au stockage des informations et objets.
	
	Le choix des technologies (HTML, CSS, JavaScript, Symfony, base de données SQL) a été confirmé, en fonction des exigences techniques au fil du projet et des compétences du groupe.
	
	\subsection{Phase de développement}
	Le développement a été réalisé de manière progressive : chaque fonctionnalité principale a été développée indépendamment en local dans un premier temps, puis ensuite intégrée progressivement au sein de la plateforme globale (GitHub).
	
	Nous avons mis en place conformément :
	\begin{itemize}
		\item Un module d'information permettant de consulter des informations sur la ville : Météo, lieux d'intérêts, événements ou encore restaurants.
		\item Un module de visualisation pour la gestion des profils utilisateurs et la consultation d'objets connectés ainsi que des outils/services.
		\item Un module de gestion dédié aux ajouts, modifications et suppressions d'objets/services.
		\item Un module d'administration permettant aux administrateurs de gérer utilisateurs, services et personnaliser la plateforme depuis un panel administrateur.
	\end{itemize}
	
	Le contrôle de version présent sur git à été très utile pour suivre l'évolution du projet, faciliter les retours en arrière et assurer la collaboration efficace entre les membres.
	
	\subsection{Phase de tests et de validation}
	Une fois le développement terminé, nous avons effectué une série de tests unitaires et fonctionnels à la fin de chaque fonctionnalités clé terminée, pour nous assurer que chaque module répondait aux attentes définies. Nous avons ensuite vérifié le bon fonctionnement et leur intégration au Front End et Back End.
	
	Des tests d'utilisation ont également été réalisés pour identifier d'éventuels problèmes qui pourraient survenir durant l'expérience utilisateur.
	
	\subsection{Résultats obtenus}
	À l'issue de ces différentes étapes, nous avons obtenu une plateforme fonctionnelle regroupant l'ensemble des fonctionnalités prévues.
	
	La plateforme permet aujourd'hui :
	\begin{itemize}
		\item D'accéder facilement aux informations locales.
		\item De gérer son profil et consulter les services disponibles.
		\item D'administrer l'ensemble des services et utilisateurs pour assurer la continuité de la plateforme.
	\end{itemize}
	
	Les objectifs initiaux ont ainsi été atteints, avec une solution numérique intuitive, agréable et adaptée aux besoins d'une ville ou d'un village intelligent.
	
	\newpage
	
	\section{Conclusion et perspectives}
	
	\subsection{Conclusion générale}
	
	Ce projet nous a offert l'opportunité de transformer des idées en une solution concrète, alliant accessibilité, innovation et service public. Chaque étape, de la conception à la réalisation, a exigé une approche rigoureuse et une capacité d'adaptation face aux défis techniques rencontrés.
	
	La construction d’une plateforme numérique centrée sur l’utilisateur nous a permis de mesurer l’importance de penser un projet dans sa globalité : depuis le début de programmation et de répartition des tâches jusque à l'organisation des fonctionnalités et la qualité du code. Cette expérience a renforcé nos compétences techniques tout en développant notre capacité à collaborer efficacement au sein d'une équipe projet.
	
	En définitive, ce travail illustre combien la technologie, lorsque elle est mise au service des citoyens, peut améliorer le quotidien et la qualité de vie d'une ville ainsi que d'ouvrir la voie à de nouvelles perspectives pour des villes et villages plus intelligents, plus connectés et plus humains.
	
	\subsection{Perspectives d'amélioration}
	
	Même si les résultats obtenus sont globalement satisfaisants par rapport aux objectifs qu'on s'était fixés par rapport aux attendus du projet, plusieurs axes d'amélioration pourraient être envisagés pour aller encore plus loin :
	
	\begin{itemize}
		\item \textbf{Mieux optimiser les performances} : En analysant ce qui pourrait ralentir le système, on pourrait rendre l'ensemble encore plus rapide et plus fluide.
		
		\item \textbf{Renforcer la sécurité} : En ajoutant des protections supplémentaires, notamment au niveau de la gestion des accès et des données, on s'assurerait que le projet soit vraiment fiable sur le long terme.
		
		\item \textbf{Rendre l’interface encore plus intuitive} : Travailler sur l'apparence et la facilité d’utilisation rendrait l’expérience encore plus agréable pour les utilisateurs.
		
		\item \textbf{Ajouter de nouvelles fonctionnalités} : En écoutant les besoins des utilisateurs, on pourrait développer des outils supplémentaires qui rendraient le projet encore plus complet.
		
		\item \textbf{Tester à plus grande échelle} : Déployer la solution dans un environnement plus large permettrait de vérifier si elle tient vraiment la route quand il y a plus de monde ou plus de données.
		
		\item \textbf{Assurer un suivi régulier} : Mettre en place un vrai suivi technique permettrait de corriger rapidement les problèmes et de continuer à améliorer le projet sur la durée.
	\end{itemize}
\end{document}
